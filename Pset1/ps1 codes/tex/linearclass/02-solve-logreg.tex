\item \subquestionpoints{5} \textbf{Coding problem.}
Follow the instructions in \texttt{src/linearclass/logreg.py} to train a
logistic regression classifier using Newton's Method.
Starting with $\theta = \vec{0}$, run Newton's Method until the updates to
$\theta$ are small: Specifically,  train until the first iteration $k$ such
that $\|\theta_{k} - \theta_{k-1}\|_1 < \epsilon$, where
$\epsilon = 1\times 10^{-5}$. Make sure to write your model's predicted probabilities on
the validation set to the file specified in the code.

Include a plot of the \textbf{validation data} with $x_1$ on the horizontal axis and $x_2$ on the vertical axis.
To visualize the two classes, use a different symbol for examples $x^{(i)}$
with $y^{(i)} = 0$ than for those with $y^{(i)} = 1$. On the same figure, plot the decision boundary
found by logistic regression (i.e., line corresponding to $p(y|x) = 0.5$).

\textbf{Note:} If you want to print the loss during training, you may encounter some numerical instability issues. Recall that the loss function on an example $(x,y)$ is defined as
$$y\log(h_{\theta}(x)) +  (1 - y)\log(1 - h_{\theta}(x)),$$
where $h_\theta(x)=(1+\exp(-x^\top \theta))^{-1}.$ Technically speaking, $h_{\theta}(x)\in(0,1)$ for any $\theta,x\in\R^{d}.$ However, in Python a real number only has finite precision. So it is possible that in your implementation, $h_{\theta}(x)=0$ or $h_{\theta}(x)=1$, which makes the loss function ill-defined. A typical solution to the numerical instability issue is to add a small perturbation. In this case, you can compute the loss function using 
$$y\log(h_{\theta}(x) + \epsilon) +  (1 - y)\log(1 - h_{\theta}(x) + \epsilon),$$
instead, where $\epsilon$ is a very small perturbation (for example, $\epsilon=10^{-5}$).
